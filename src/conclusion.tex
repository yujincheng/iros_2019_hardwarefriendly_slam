\label{sec:conclusion} 

We propose a hardware-software co-design DSLAM system with the help of Xilinx Zynq MPSoC and Deephi DPU. We optimize two essential components of the DSLAM system,  $1)$Visual Odometry(VO) and $2)$ Place Recognition on the embedded system. From the aspect of calculation, the vectorization and projection operations on the PS side of Zynq MPSoC needed by NetVLAD is the bottleneck in doing more frequent place recognition. From the aspect of communication, the data traffic for inter-robot relative pose estimation consumes the most communication resources. The accelerator for vectorization based on FPGA and more data-efficient inter-robot pose estimation method could be researched in future work.