\label{sec:introdutction}
In recent years, with the development of the hardware and algorithms, the capabilities of a single agent have been greatly improved.
To futher expand the capabilities of intelligent robots, using several robots can accelerate many tasks, such as localization, exploration and mapping.
As simultaneous localization and mapping(SLAM) is an essential component in many tasks, it is important to do SLAM across different robots in many multi-agent applications. 
The camera is a widely used snesor in SLAM for its rich information and low cost. 
However, in many scenarios, communication is limited, so that there is no a server or a agent can stably collect all of the visual data from each robot.

Therefore, to reduce computational requirements, the previous work \cite{Cieslewski:20187ee} proposes a data-efficient decentralized SLAM(DSLAM) system and makes improvements in three typical components in DSLAM system:
\begin{itemize}
\item Using ORB-SLAM \cite{orbslam} in stereo configuration as the visual odometry algorithm which provides basic intra-robot position estimation.
\item Using NetVLAD \cite{Arandjelovic:2017997} algorithm to do place recognition which relates the current observation to previous scenes and other robots.
\item Using distributed Gauss-Seidel algorithm \cite{parallel_distributed} as the optimization back-end which optimizes the intra-robot position and fuses the inter-robot locations and maps.
\end{itemize}

Both ORB-SLAM and NetVLAD require tremendous computation and storage resources, and thus, the deployment of DSLAM on embedded system is challaged by the limited resources and power supply.

The NetVLAD algorithm based on VGG-16 model \cite{vgg16} consumes more than 80G operations for a single $300 \times 300$ input image. Each operation means a addition or a multiplication. It is very challaging to deploy the NetVLAD on a tranditional embedded hardware platform. 
As to ORB-SLAM, the algorithm need to calculate the feature point and corresponding descriptors for each frame from camera.

To make the DSLAM system more energy efficient and hardware friendly, we propose a novel hardware-software codesign DSLAM framework with 